Using polychromatic light, the interferogram at the $n^{th}$ output can be expressed as a function of the optical path difference, x, as follow :
\begin{equation}\label{eq:poly}
In(x) = \int_{-\infty}^{+\infty}I_A(\sigma)\kappa_A(\sigma)+I_B(\sigma)\kappa_B(\sigma)+2\sqrt{I_A(\sigma)\kappa_A(\sigma)I_B(\sigma)\kappa_B(\sigma)}\left| \mu_{AB}(\sigma)\right| cos(\phi_{AB}(\sigma)-2\pi\sigma x) d\sigma
\end{equation}
where $\kappa_i$ relates the transmission from the $i^{th}$ input, $I_i$ the normalized intensity at the $i^{th}$ input, $\left| \mu_{AB}(\sigma)\right| =\left| \mu_{AB}(\sigma)^{inst}\right|\left| \mu_{AB}(\sigma)^{obj}\right| $ the visibility of the interferogram and $\phi_{AB}(\sigma)=\phi_{AB}^{inst}(\sigma)+\phi_{AB}^{obj}(\sigma)$ the phase of the interferogram.
In order to build a V2PM matrix which is independent of the spectrum of the source, it is needed to assume that the spectrum is "flat" within the considered bandwidth. Thus the V2PM matrix will be valid only for quasi-monochromatic light (i.e. a small bandwidth). Doing that the terms $I_i$ are no more wavelength dependant. Eq. \ref{eq:poly} becomes :
\begin{equation}\label{eq:poly2}
In(x) = t_{A}\int_{\sigma}I_Ad\sigma+t_B\int_{\sigma}I_Bd\sigma+2\sqrt{I_AI_B}\int_{\sigma}\sqrt{\kappa_A(\sigma)\kappa_B(\sigma)} \left| \mu_{AB}(\sigma)\right| cos(\phi_{AB}(\sigma)-2\pi\sigma x)d\sigma
\end{equation}
In which $t_i=\frac{\int_{\sigma}I_i(\sigma)\kappa_i(\sigma)}{\int_{\sigma}I_i(\sigma)}$. As our assumption lead us to be limited to quasi-monochromatic light, the visibility should also be relatively independent of the wavelength, as well as the phase if the dispersion of the instrument is negligible. Thus Eq.\ref{eq:poly2} becomes :
\begin{equation}\label{eq:poly3}
In(x) = t_{A}\int_{\sigma}I_Ad\sigma+t_B\int_{\sigma}I_Bd\sigma+2\sqrt{I_AI_B}\left| \mu_{AB} \right|\int_{\sigma}\sqrt{\kappa_A(\sigma)\kappa_B(\sigma)}  cos(\phi_{AB}-2\pi\sigma x)d\sigma
\end{equation}

In order to use the same formalism as in the monochromatic case, we use the so called photocorrection :
$$
\tilde{I_n} = I_n -t_{A}\int_{\sigma}I_Ad\sigma-t_B\int_{\sigma}I_Bd\sigma
$$
\begin{equation}\label{eq:photocorpoly}
V_{AB} =\frac{2\sqrt{I_AI_B}\left| \mu_{AB} \right|\int_{\sigma}\sqrt{\kappa_A(\sigma)\kappa_B(\sigma)}  cos(\phi_{AB}-2\pi\sigma x)d\sigma}{2\sqrt{t_{A}\int_{\sigma}I_A t_B\int_{\sigma}I_B}}
\end{equation}

In that case the visibility function is no longer a cosine, but it can be seen as the Fourier transform of the spectral response of the component (in the case of the a flat spectrum signal used at the input). Thus the visibility function is now something like a cardinal sine, and the narrower the bandwidth, the closest to a cardinal sine it gets.

To use the same V2PM format, the amplitude of visibility is taken at the maximum of this interferogram, where $V_{AB} \approx \left|\mu_{AB}\right|$. The phase is also taken at the position $x_0$ of this maximum and reduced using the formula $\Phi_{AB} \approx 2\pi\sigma_0 x_0$ where $\sigma_0$ is the middle range wave number of the input signal. Taking the nearest point to the "0" OPD is important as will be further explained in the next section.  

