The work presented in this report was aimed at optimising the Zig-Zag DBC and experimentally demonstrating it's capabilities to combine the light from 4 individual telescopes at the same time. Because of space and time restriction this work is not all the work done during this internship. Not included in this report are simulations of tri-couplers, asymmetric-couplers and experimental characterizations of asymmetric-couplers and MMIs for nulling purpose. 

In the first part the DBC has been optimised at $\lambda=3.4 \mu m $ regarding the condition number of the V2PM. A condition number down to 6 was obtained. Moreover the problem of determining the flux truly guided through one output and the presence of harmonics due to the neighbouring WG has led to minimizing the integrating area and developing a design with a fan-out. This fan-out led to 3\% of bending losses but considerably increased the usable throughput. 
Simulations have demonstrated the  ability of both design to retrieve the visibility and phase of the simulated punctual source of 1 baseline at a time for signal spectral bandwidth up to 70nm with an accuracy of $\pm 5\%$ on the visibility and $\pm 0.05 rad$ on the phase. When it comes to combining the 4 beams at the same time the design with fanout has shown better accuracy (approximately two times) on the retrieved parameters at 20nm bandwidth. To the best of my knowledge these differences are mostly linked to the transmission coefficient $\kappa$ being highly wavelength dependant leading to an apparent wavelength of the output signal being different from $3.4 \mu m$. An other limiting factor to better retrieved visibility is the "0 OPD" being at different input OPD for each output and this is mostly impacting the higher bandwidth because of smaller coherence length. Accuracy of $\pm 5\%$ on the retrieved visibilities and $\pm 0.08 rad$ on the retrieved phases has been obtained with the component with fanout and a spectral bandwidth of the input signal of 20nm. Further simulation at larger bandwidth should be done to confirm or invalidate the superiority of the component with fanout.

In the second part the experimental demonstration of the capabilities of the DBC was done with a signal spectral bandwidth of 70nm centred at $3.745 \mu m$. Condition number down to 25 was obtained and errors ranging from 10 to 20\% on the retrieved visibility and 0.08 to 0.4 rad on the retrieved phases has been obtained (combining two beams at a time). High sensibility to the coupling has been experienced and those errors are mostly due to the experimental setup, especially the delay-lines being not reproducible in their movement. When combining the 4 beams at the same time, far lower accuracy was obtained (similarly to what was obtained in simulation). Further work with the previously optimized component with fanout would have to be done.

\paragraph{Further work :}
If would have to continue this work I would identify few axes of work.

The first one would be to continue simulations using the 4 beams for bandwidth up to 70 nm in order to validate or not the superiority of the design with fan-out.

The second one would be to get the transmission coefficients as independent as possible to the wavelength. This could results in better accuracy in retrieving the visibility and phase. Moreover trying to get the optical path difference between each output signal (i.e getting the maximum of their interferogram at the same input OPD) could highly increase the accuracy and the 'bandpass' of the component. 
To do the first one, one could adapt the work done on asymmetric couplers but doing few short portion of the waveguides at a different optical index and different position. To the best of my knowledge It will not be possible to strictly adapt the same procedure done with asymmetric couplers to the DBC.
To do the second one I tested using an array of 17 waveguides, this worked but the condition number of the V2PM sharply increased (as suggested by \cite{minardi1}). 

The results obtained experimentally with the component showed lower contrast than expected from the simulation. This is probably due to high birefringence induced by the ultra-fast laser inscription method. Measurements on the DBC with a polariser would have to be done in order to verify that. Moreover the dependence of the V2PM matrix regarding the polarisation could be an interesting thing to study as well as the accuracy on the retrieved parameters.

The DBC present interesting and promising capabilities as it demonstrated high contrasts both in simulation and experimentally, and good accuracy on the retrieved astronomical parameters. 