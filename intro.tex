Since antiquity and down to our time, astronomers always tried to see further and further in space requiring more and more sensitive instruments. Increasing the telescope diameter is one way to reach higher angular resolution but in the same time make it more sensitive to atmospheric turbulence. Therefore even with the recent progress in adaptative optics, today's largest telescopes can only can only resolve few of the brightest and nearest stars. 

Using individual telescopes to form an interferometer, the resolution is determined by the distance between the telescopes. Until recently the instruments combining the light from individual telescopes were bases on costly and cumbersome bulk optics. The recent advances in manufacturing integrated-optics and especially in laser processing have resulted in new instruments that are operational on sky and delivering higher quality results. 

The purpose of this work is to optimise and characterise the performance of \gls{io} beam combiners and especially one promising type of component called \gls{dbc}. Allowing to retrieve the astronomical parameters without scanning the interferogram these components could allow to observe fast varying objects. 

This report is organised in three parts. In the first part I will present the motivation and the basis of astronomical interferometry. In the second part will be presented the simulation results of the \gls{dbc} as well as its optimisation. In the last part will be discussed the experimental characterisation of asymmetric couplers, \gls{mmi} and of the \gls{dbc}.
