\documentclass[a4paper,twoside,11pt]{article}
\usepackage[english]{babel}
\usepackage[utf8]{inputenc}
\usepackage[T1]{fontenc}
\usepackage{lmodern}

\usepackage{todonotes}
\usepackage{natbib}
\usepackage{url}
\usepackage{amsmath}
\usepackage{graphicx}
\graphicspath{{images/}}
\usepackage{parskip}
\usepackage{fancyhdr}
\usepackage{vmargin}
\setmarginsrb{3 cm}{2.5 cm}{3 cm}{2.5 cm}{1 cm}{1.5 cm}{1 cm}{1.5 cm}
%\setmarginsrb{30mm}{20mm}{20mm}{20mm}{0pt}{0mm}{0pt}{0mm}

\pagestyle{fancy}
\fancyhead{}
\fancyfoot{}
\fancyhead[LE,RO]{\thepage}
\fancyhead[LO,RE]{\rightmark}

\title{Simulation and characterization of integrated optics beam combiners for astrointerferometry}								% Title
\author{Thomas Poletti}								% Author
\date{\today}											% Date

\makeatletter
\let\thetitle\@title
\let\theauthor\@author
\let\thedate\@date
\makeatother


\renewcommand{\headrulewidth}{0pt}
\renewcommand{\footrulewidth}{0pt}

\graphicspath{ {images/} }

\newcommand\frontmatter{%
    \clearpage
  %\@mainmatterfalse
  \pagenumbering{roman}}

\newcommand\mainmatter{%
    \clearpage
 % \@mainmattertrue
  \pagenumbering{arabic}}

\newcommand\backmatter{%
  \if@openright
    \clearpage
  \else
    \clearpage
  \fi
 % \@mainmatterfalse
   }

   
\usepackage[linktoc=all]{hyperref}
\hypersetup{
    colorlinks,
    citecolor=black,
    filecolor=black,
    linkcolor=black,
    urlcolor=black
}

\usepackage[nottoc]{tocbibind}

\begin{document}

\listoftodos
%%%%%%%%%%%%%%%%%%%%%%%%%%%%%%%%%%%%%%%%%%%%%%%%%%%%%%%%%%%%%%%%%%%%%%%%%%%%%%%%%%%%%%%%%

\begin{titlepage}

	\begin{minipage}{0.5\textwidth}
		\begin{flushleft} 
    %\vspace*{0.5 cm}
    \includegraphics[scale = 0.6]{phelma.png}\\[1.0 cm]	% University Logo
			\end{flushleft}
			\end{minipage}~
			\begin{minipage}{0.5\textwidth}
            
			\begin{flushright} 
    
    %\vspace*{0.5 cm}
    \includegraphics[scale = .9]{university.png}\\[1.0 cm]	% University Logo
		\end{flushright}
        
	\end{minipage}\\[3 cm]
	
    \centering
    \vspace*{0.5 cm}
	\rule{\linewidth}{0.2 mm} \\[0.4 cm]
	{ \huge \bfseries \thetitle}\\
	\rule{\linewidth}{0.2 mm} \\[1.5 cm]
    \textsc{\Large I. Physikalisches Institut \\
Universität zu Köln}\\[5 cm]	% University Name
	%\textsc{\Large Grenoble INP-Phelma}\\[0.5 cm]				% Course Code

	
	\begin{minipage}{0.4\textwidth}
		\begin{flushleft} \large
			\emph{Author :} \\
			Thomas Poletti\\
  %           13103057\\
          M1 Physics and NanoSciences\\
          School-year 2017-2018\\
   %         Semester\\
			\end{flushleft}
			\end{minipage}~
			\begin{minipage}{0.4\textwidth}
            
			\begin{flushright} \large
   			\emph{Supervisor:}\\
			Pr. Dr. Lucas Labadie\\
            labadie@ph1.uni-koeln.de\\
		\end{flushright}
        
	\end{minipage}\\[2 cm]
	
    
    
	
\end{titlepage}

\frontmatter
\begin{abstract}
   abstract-text
\end{abstract}
\renewcommand{\abstractname}{Résumé}
\newpage
\begin{abstract}
   Résumé ici
\end{abstract}

\newpage
\tableofcontents
%\addcontentsline{toc}{section}{\listfigurename}
%\addcontentsline{toc}{section}{\listtablename}
\listoffigures
\listoftables

\clearpage

\mainmatter
\section{Motivation and scientific background}

\todo[inline]{- quoi qu’est-ce l’interferometrie, intérêt par rapport à l’observation par un télescope mono-pupille en terme
de résolution angulaire
- nulling, contraintes
- schema hi-5, explications
- VLTI
- bandes d’observation
- unités astro
- Bases de la BeamPropMethod}

\section{Simulation of Discret Beam Combiner}
    \subsection{Monochromatic light}
        \subsubsection{Mathematical formalism}
        \subsubsection{Influence of geometrical parameters}
        \subsubsection{Retrieving the visibility function}
    \subsection{Polychromatic light}
        \subsubsection{Mathematical formalism}
        \subsubsection{Influence of the bandwidth}
        \subsubsection{Retrieving the visibility function}
        
\section{Laboratory characterization of beam combiners}
    \subsection{Asymmetric couplers}
    \subsection{MMI}
    \subsection{Discrete Beam Combiner}

\section{Conclusion and Further-work}

\backmatter
\appendix
\section{Appendix}
    \subsection{Appendix 1}
\end{document}    
